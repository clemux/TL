\documentclass[12pt, a4paper]{report}

\usepackage[utf8]{inputenc}
\usepackage[T1]{fontenc}
%\usepackage[french]{babel}
\usepackage{amsmath}
\usepackage{amssymb}
\usepackage{stmaryrd}
\usepackage{tikz}

\title{Théorie des langages}
\author{Licence d'informatique}
\date{Université de Strasbourg}

\begin{document}

\maketitle

\part{Langages}

\chapter{Introduction}

\section*{Quelques définitions}

\paragraph{Langage} On appelle \textit{langage} tout ensemble de mots.
Il existe deux façons de définir un langage : par extension ou par intention Le mot vide est noté $\Sigma$

\paragraph{Alphabet} On appelle \textit{alphabet} tout ensemble fini (non vide)
de symboles.


\paragraph{Grammaire} 

Pour les langages de programmation, un programme peut être vu comme un
mot faisant partie de ce langage. Le compilateur vérifie ensuite dire
que ce mot est conforme au langage, à ce qu’il attend. On met donc des
mécanismes d’analyse syntaxique, on parle de \textit{grammaire}.


\paragraph{Exemple} langage binaire

Alphabet : $\{0, 1\}$

Mots : 0, 1, 01, 00001.


\begin{center}
\begin{tabular}{|c|c|c|}
\hline
classes de langages & reconnaissance d'un mot & génération d'un mot \\
\hline
\hline
langage régulier  & automates finis & grammaire régulière \\ \hline
langage algébrique & automates à pile & grammaire algébrique \\ \hline
langage récursif & machine de Turing & grammaires sans contrainte \\
\hline
\end{tabular}
\end{center}

Ces classes de langage sont présentées du plus restreint au plus large, ou
plus formellement :
\[
\text{langages réguliers}
\;\subseteq\text{langages algébriques}
\;\;\subseteq\text{langages récursifs}
\;\;\;\subseteq\text{tous les langages}
\]

\section{Ensemble, relation, fonction et langage}


Un \textit{ensemble} est une collection non-ordonnée d'objets qui sont
appelés les \textit{éléments} de l'ensemble.

Les ensembles ont été formellement définis suite au \textit{paradoxe
  de Russel} .

\subsection{Exemple}

\[
I = {1, 2, 3, 5, 8, 10}
\]


\subsection{Opérations sur les ensembles}

Soient $A$ et $B$ deux ensembles.

\paragraph{Union}
\[ A \cup B = \{ e | e \in A \textrm{ ou } e \in B \} \]

\paragraph{Intersection}
\[ A \cap B = \{ e | e \in A \textrm{ et } e \in B \} \]

\paragraph{Différence}
\[ A \backslash B = \{ e | e \in A \textrm{ et } e \not\in B \} \]

\paragraph{Différence symétrique}
\[ A \Delta B = (A \backslash B)\cup (B \backslash A) \]

\paragraph{Ensemble des parties}
\[ P(A) = { C | C \subseteq A} \]

\paragraph{Produit cartésien}
\[ A \times B = {(a, b) | a \in A \text{ et } b \in B} \]

\subsection{Lois de Morgan}
Soient $A, B, C$ des ensembles. Alors
\[
A\backslash(B \cup C) = (A\backslash B=\cap(A\backslash C) \\
A\backslash(B \cap C) = (A\backslash B=\cup(A\backslash C)
\]

\paragraph{Remarques :} soient $A, B$ deux ensembles. Alors :
\[
A=B \Leftrightarrow A\subseteq B \text{ et } B\subseteq A
\]

Pour des $n$-uplets, l'égalité devient a :
\[
(e_1,e_2,\dots,e_n)=(f_1,f_2,\dots,f_n)\Leftrightarrow e_1=f_1,e_2=f_2,
\dots,e_n=f_n
\]

\subsection{Définition}


Soit $A$ un ensemble et $\Pi \subseteq \mathcal{P}(A)$

$\Pi$ est dit une partition de $A$ si et seulement si :
\begin{itemize}
\item $A = \bigcup_{k\in K} \text{ ou }\Pi=\{B_k \mid k\in K\}$
\item $B_i\bigcap B_j = \emptyset \; \forall i, j\in K \mid i\neq j$
\item $B_i \neq \emptyset$
\end{itemize}

Exemple : $A=\{1,2,3,4,5\}, \Pi=\{\{1,3\},4,\{2,5\}\} = \{B_1,B_2,B_3\}$

\subsection{Relations}

Soient $E_1$, $E_2$, $\dots$, $E_n$, $n$ ensembles. Une relation $R$
$n$-aire (d'arité $n$) sur $E_1$, $E_2$, ..., $E_n$ est un
sous-ensemble de $E_1 \times E_2 \times \dots \times E_n$.


Autrement dit $R \subseteq E_1 \times E_2 \times \dots \times E_n$.


Pour $n = 2$, R est dite une relation binaire.

\paragraph{Notation} Si $E = E_1 = E_2 = \dots = E_n$, alors $E_1 \times \dots
\times E_n$ est noté $E^{n}$.



\subsection{Relations fonctionelles}

Soit $R$ une relation $n$-aire sur $E_1\times E_2\times\dots
E_{n-1}\times E_n$ avec $n\geq 2$. Si, pour tout
$(e_1,e_2,\dots,e_{n-1})\in E_1\times E_2\times\dots E_{n-1}$, il
existe \textit{un et un seul} $e_n\in E_n$ tel que
$(e_1,e_2,\dots,e_{n-1},e_n) \in R$.

On dit alors que $R$ est une relation fonctionnelle relativement à
$E_1\times E_2\times\dots E_{n-1}$. On utilisera la notation : $e_n =
f_R(e_1,\dots,e_{n-1})$ qui est l'unique élément tel que
$(e_1,e_2,\dots,e_{n-1},e_n)\in R$.

De plus, $f_R$ est appelé la fonction associée à la relation
fonctionnelle $R$. $E_1\times E_2\times\dots E_{n-1}$ est appelé le
domaine (ou ensemble de départ) de $f_R$ et $E_n$ est appelé
l'ensemble d'arrivée de $f_R$.

\paragraph{Notation} soit $R$ une relation fonctionne sur $E_1\times\dots\times
E_{n-1}\times E_n$. On note :
\[
f_R : E_1\times\dots\times E_{n-1} \longmapsto E_n
\]

\paragraph{Définitions} soient $A, B$ deux ensembles et $f_R : A\mapsto B$.

\begin{itemize}
\item $f_R$ est dite injective (une injection) si $\forall a, a' \in
  A$
\[a\neq a' \Rightarrow f_R(a) \neq f_R(a')\]

\item $f_R$ est dite surjective (une surjection) si $\forall b\in B$,
  il existe $a\in A$ tel que $b = f_R(a)$

\item $f_R$ est dite bijective (une bijection) si $f_R$ est à la fois
  injective est bijective.

\item Lorsque la fonction $f_R$ est bijective, alors $f_R^{-1}$ est la
  fonction associée à la relation $R^{-1} = \{(b,a) \mid (a,b) \in
  R\}$. En particulier, $R^{-1}$ est fonctionelle relativement à
  $B$. Autrement dit :
\[f_R^{-1} = f_{R^{-1}}\]
Dans ce cas, $f_{R^{-1}}$ est bijective.

\item De plus, $\forall (a, b) \in A\times B$ : 
\[b=f_R(a) \leftrightarrow a = f_R^{-1}(b)\]
Si $f_R$ est bijective, alors $\big((f_R)^{-1}\big)^{-1} = f_R$

\item Soient $R_1 \subseteq A\times B$ et $R_2 \subseteq B\times C$
  deux relations binaires.
\[R_1 \circ R_2 = \{(a,c) \in A \times C \mid \exists b\in B
\text{ tel que }(a,b)\in R_1 \text{ et } (b,c)\in R_2\} \]
$R_1 \circ R_2$ est la composition de $R_1$ et de $R_2$.
\end{itemize}

\paragraph{Propriété :} soit $R$ une relation fonctionnelle sur $A\times B$
telle que $f_R$ est bijective. Alors :

\begin{itemize}
\item $R\circ R^{-1} = \{(a,a)\mid a\in A\}$
\item $R^{-1}\circ R = \{(b,b)\mid b\in B\}$
\end{itemize}

\subsection{Relations fonctionnelles particulières}
Soir $R$ une relation binaire sur $A^2 (R\subseteq A^2)$.

\begin{itemize}
\item $R$ est dite réflexive si $\forall a\in A, (a,a)\in R$

\item $R$ est dite symétrique si $\forall a, b\in A, (a,b)\in R
  \Rightarrow (b,a)\in R$

\item $R$ est dite transitive si $\forall a, b, c\in A, (a,b)\in R$ et
  $(b,c)\in R \Rightarrow (a,c)\in R$

\item $R$ est une relation d'équivalence si $R$ est à la fois
  réfléxive, symétrique et transitive.

\item $R$ est dite antisymétrique si $\forall a, b\in A, (a,b)\in R$
  et $a\neq b \Rightarrow (b,a)\not\in R$
\item $R$ est une relation d'ordre si $R$ est à la fois réflexive,
antisymétrique et transitive.
\item $R$ est une relation d'ordre total si $R$ est une relation d'ordre et pour
tout $a, b\in A$, si $(a,b) \not\in R$, alors $(b,a)\in R$
\end{itemize}

\subsection{Représentation de relations}

Soit $A = \{1,2,3\}$ un ensemble, et la relation $R = \{(1,2),(1,3),(2,3)\}$
une relation sur $A$. On peux représenter cette relation sous plusieurs formes :

\paragraph{Forme matricielle}
%\[
%M = \bordermatrix{~ & 1 & 2 & 3 \cr
%                  1 & 0 & 1 & 1 \cr
%                  2 & 0 & 0 & 1 \cr
%                  3 & 0 & 0 & 0 \cr}
%\]

\paragraph{Graphes}
% représentation en graphe
%R = (1,2),(1;3),(2,3)
%1 -> 2
%1 -> 3
%2 -> 3
%%% mux> ça compile pas, j'ai commenté
% \begin{tikzpicture}
% \tikzstyle{vertex}=[circle,fill=black!25,minimum size=20pt,inner sep=0pt]
% \tikzstyle{edge} = [draw,thick,->]
% \tikzstyle{weight} = [font=\small]
% \begin{figure}
% \begin{tikzpicture}[scale=1.8, auto,swap]
%    % draw the vertices
% \foreach \pos/\name in {{(0,2)/a}, {(3,2)/b}}
%     \node[vertex] (\name) at \pos {$\name$};
% % Connect vertices with edges and draw weights
% \foreach \source/ \dest /\weight in {a/b/1, b/a/2}
%     \path[edge] (\source) -- node[weight] {$\weight$} (\dest);
 
% \end{tikzpicture}
% \end{figure}


\paragraph{Notations}

Soit $R$ une relation d'équivalence $A \times A$ ($A^{2}$) et soit $a
\in A$.

\begin{itemize}

%%%% CHECKME vérifier les accolades
\item $[a]_\{R\} = \{ b \in A \mid (a,b) \in R\} \}$ est la classe de $A$
relative à la relation d'équivalence $R$

\item $A/R = {[a]_{R} | a \in A}$ est le quotient de $A$ par $R$
\end{itemize}

\paragraph{Propriétés}

\begin{itemize}

\item Soit $R$ une relation d'équivalence par $A \times A$. Alors
  $A/R$ est une partition de l'ensemble $A$.

\item Réciproquement si $\Pi$ est une partition de $A$ alors il existe
  une et une seule relation d'équivalence $R_{\Pi}$ tel que $\Pi =
  A/R_{\Pi}$.

\end{itemize}

\paragraph{Preuve :}
\begin{itemize}

\item $\forall a\in A, a\in [a]_R$ car $R$ est réflexive. Donc $A =
  \bigcup_{a\in A}[a]_R$

\item Soient $a, a\in A$. Montrons que si $[a]_R\cap [a']_R
  =\emptyset$, alors $[a]_R=[a']_R$
\end{itemize}

\paragraph{Relations binaires particulières}


Soit $b \in [a]_{R} \cap [a']_{R}$.


Alors $(a, b) \in R$ et $(a', b) \in R$


Comme $R$ est symétrique alors $(b, a') \in R$ et comme $R$ est
transitive alors $(a, a') \in R$.

Et donc $\forall c \in [a']_{R}$ on a $(a', c) \in R$ et donc par
transitivité $(a,c) \in R$ et donc $c \in [a]_{R}$.


Soit $\Pi$ une partition de $A$

$\pi = { E_i \mid i \in I }$

\begin{enumerate}
\item $A = \bigcup_{i \in I) E_i}$
\item $E_1 \cap E_j = \emptyset$ pour tout $i, j \in I$ et $i\neq j$
\item $E_i \neq \emptyset$ pour tout $i \in I$
\end{enumerate}

$\Pi \leftrightarrow R_{\Pi} \rightsquigarrow (a,b)\in R_{\Pi} \text{ ssi } \exists i \in I \text{ tel que } a, b \in E_i$


\paragraph{Définition}

Soit $R$ une relation binaire sur un ensemble $A (R \subseteq A^2)$ la suite $x_0, x_1, ..., x_n$ d'éléments de $A$ est dite un chemin de $R$ si $(x_i, x_{i+1}) \in \forall 0 \leq i < n$.

\begin{itemize}
\item $x_0$ est appelé le début du chemin $C$
\item $x_n$ est appelé la fin du chemin $C$
\item $n$ est appelé la longueur du chemin $C$
\end{itemize}

\section{Cardinalité d'un ensemble}

\subsection{Définitions}

Deux ensembles $E_1$ et $E_2$ sont dits de même cardinal s'il existe une bijection
$f : E_1 \mapsto E_2$

Soient $E_1$ et $E_2$ deux ensembles. On dit que le cardinal de $E_1$ est strictement inférieur
à celui de $E_2$ s'il existe une fonction injective $f : E_1 \mapsto E_2$ et il n'existe pas de
bijection de $E_1$ à $E_2$

Un ensemble E est dit fini s'il existe un entier $n \in \mathbb{N}$ tel qu'il y a une bijection
$f : \llbracket 1, n \rrbracket \mapsto E$. % c'est quoi l/r brace? -Crochets?

Dans ce cas on dit que $E$ est de cardinal $n$. % De plus n est unique (hein?)

Un ensemble E est dit infini dénombrable s'il a le même cardinal que $\mathbb{N}$ % |N
(il existe une bijection $f : N \mapsto E$).

\paragraph{Exemples}
\begin{itemize}


\item{$\mathbb{Z}$ est un ensemble infini dénombrable}

\item{$\mathbb{Q}$ est un ensemble infini dénombrable}

\item{$\mathbb{N}$ est un ensemble infini dénombrable}

\item{$\mathbb{N}^*$ est un ensemble infini dénombrable}


\item{:$\mathbb{N}^2  \rightarrow \mathbb{N}$

f(i,j) = $(\sum_{k=1}^{i+j} k) + i = (i+j)(i+j+1)/2 + i$ est une bijection de $\mathbb{N}^2  \text{ dans } \mathbb{N}$}

\item{$Q^{+} = {i/j | (i,j) \in N\times N^{*} \text{ et }\text{ pgcd}(i, j) = 1}$}

\item{$Q = Q^{+} \cup (- Q^{+})$}

\item{$Q^{+} = \{ -q | q \in Q^{+}\}$}

\end{itemize}



\subsection{Propriétés}

Soit E un ensemble alors il n'existe pas de bijection entre $E$ et $\mathcal{P}(E)$.

En fait, le cardinal de $E$ est strictement inférieur au cardinal de $\mathcal{P}(E)$.


\paragraph{Preuve}

Soit $f : E \mapsto \mathcal{P}(E), f(e) \mapsto {e}$

Par construction, $f$ est injective.

Montrons qu'il n'y a pas de bijection entre $E$ et $\mathcal{P}(E)$.

Suppons qu'il existe une bijection $g :
E \mapsto \mathcal{P}(E)$.
Posons $A = { e \in E \mid e \not\in g(e)} \in P(E)$.

Comme $g$ est symétrique et $A \in \mathcal{P}(E)$ alors il exise $e_0 \in E$ tel que $g(e_0) = A$.

Deux cas sont possibles.

\begin{enumerate}
\item $e_0 \in A$ implique $e_0 \not\in g(e_0)$ donc $e_0 \not\in A$ ce qui est absurde.
\item $e_0 \not\in A$ ce qui implique $e_0 \in g(e_0)$ donc $e_0 \in A$ ce qui est absurde.
\end{enumerate}

Donc l'hypothèse de départ est absurde, et donc il n'y a pas de bijection entre $E$ et $\mathcal{P}(E)$.



$R$ et $R^2$ ont le même cardinal 
$R$ et $P(N)$ ont le même cardinal 

% mux> < is weird <, plz adjust for actual symbol  (this english is no real english either)
% mux> c'est \prec je crois
$E < \mathcal{P}(E) < \mathcal{P}(\mathcal{P}(E)) < .........$
$N < \mathcal{P}(N) < \mathcal{P}(\mathcal{P}(\mathbb{N})) < ...$

% what is exercise ?
% en bas à gauche dude




%
$f : N^2 \mapsto N$
$f(i,j) = (\sum^{i+j}_{k=1}) + i$

Montrons que $f$ est surjective:

$f(0, 0) = 0$

Soit $n\leq 0$. Supposons qu'il existe $(i,j) \in \mathbb{N}^2$ tel que $f(i,j) = n$
et montrons que cela implique qu'il existe $(i',j')\in \mathbb{N}^2$ tel que $f(i,j) = n+1$
\[
f(i,0) = n = \Sigma^i_{k=1)}k+i
i'=0, j'=i+1
f(i',j') = f(0, i+1) = \Sigma^{i+1}_{k=0)}k+0 = \big(\Sigma^i_{k=0)} k\big)+i+1
\qquad = f(0,i)+1 = n+1
\]

\begin{center}
\textit{faire la suite en exercice}
\end{center}

\subsection{Méthodes de raisonnement}
\begin{enumerate}
\item Principe d'induction

Soit $A\subseteq \mathbb{N}$ tel que :
\begin{enumerate}
\item $0\in A$ et
\item $\forall n \in \mathbb{N}, \text{ si } \llbracket 0,n\rrbracket
\subseteq A$ alors $n+1\in A$
\end{enumerate}
Alors : $A=\mathbb{N}$.
\paragraph{Preuve :} supposons que $A\neq \mathbb{N}$

Soit $n_0$ le plus petit élément de $\mathbb{N}\setminus A$.
Donc $n_0\neq 0$ et $\llbracket 0,n_0-1\rrbracket \subseteq A$, donc $n_0\in A$,
ce qui est absurde. Donc : $A=\mathbb{N}$.







% CHECKME: structure
%tibo : corrigé, c'est au même niveau que "Principe d'induction"
\item Principe des tiroirs et des pigeons. 

Soient T et P deux ensembles finis tels que $card(P) > card(T)$  alors il
n'existe pas de fonction injective de P dans T.

%%% EXO: démontrer

\item Principe de diagonalisation.

Soit R une relation binaire sur un ensemble E ($R \subseteq E^2$)

Notation : Pour tout $e \in E$, posons R(e) = ${e' \in E | (e, e') \in R}$.

$D(R) = {e \in E | (e, e') \not\in R}$.

Alors $D(R) \neq R(e)$ pour tout $e\in E$.

Preuve : Supposons qu'il existe $e \in E$ tel que $ D(R) = R(e)$.

Deux cas sont possibles :
\begin{enumerate}
\item $e \in R$ ce qui implique $(e, e) \in R$ ce qui implique que $e \not\in 
D(R)$ et donc $e \not\in R(e)$ ce qui est absurde.

\item $e \not\in R(e)$ ce qui implqiue que $(e,e) \not\in R$ ce qui implique $e
\in D(R)$ et donc $e \in R(e)$ ce qui est absurde.

\end{enumerate}

Donc l'hypothèse de départ est absurde, et donc $D(R) \not\in R(e)$ pour tout $e
\in E$.
\end{enumerate}

\paragraph{Propriété}
Soit $R$ une relation binaire sur un ensemble fini $E (R\subseteq E^2)$ tel que
$n=card(E)$, et soient 
$e, e'\in e$. Si il existe un chemin $C$ dans $R$ de début $e$ et de fin $e'$,
alors il existe 
nécéssairement un chemin de début $e$ et de fin $e'$ et de longeur $l \leq n-1$.

\textbf{Preuve :}(rappel)

Soit $C = e_0, e_1, \dots, e_m$ un chemin de début $e$, de fin $e'$ et de
longueur $m$. Alors, 
deux cas sont possibles :

\begin{enumerate}
  \item $m \leq n-1$ et dans ce cas, le chemin cherché est $C$.
  \item $m \geq n \Rightarrow m+1 \geq n+1$. Donc, il existe $0\leq i < j < m$
  tel que $e_i = e_j$.
  Posons $C' = e_0, e_1, \dots, e_i, e_{j+1}, \dots, e_m$ est un chemin de
  début $e$ et de fin $e'$
  plus court que $C$.
\end{enumerate}

\paragraph{Définition}

La clôture réflexive et transitive $R^{*}$ d'une relation binaire $R$ sur un ensemble E
($R \subseteq E^2$) est la plus petite (au sens de l'inclusion) relation binaire sur $E$ contenant
$R$ ($R \subseteq R^{*}$) et qui est réflexive et transitive.

\paragraph{Remarque} $R^{*} = { (e, e') \in E^{2} | \exists \textrm {chemin dans R de début e et de fin e'}}$

\paragraph{Algorithme 1}
%%% FIXME: trouver bonne façon de présenter un algo
Donnée : $R \subseteq E^2$

Résultat : $R^{*}$ (la clôture réflexive et transitie de $R$)

Initialisation : $R^{*} = R$

Pour $i$ de 1  $n$ faire :

  Pour chaque i-uplet ($e_{j1}, e_{j2}, \dots, e_{jn}) \in E^i$ faire
     si ($e_{j1}, e_{j2}, \dots, e_{jn}$) est un chemin relativement à R alors
     $R^{*} = R^{*} \cup { (e_{j1}, e_{jn})}$


\paragraph{Algorithme 2}
%% EXO vérifier que l'algorithme est correct 

Donnée : $R \subseteq E^{2}$ ($ E = {e_1, e_2, \dots, e_n}$)

Résultat : $R^{*}$ (...)

Initialisation : $R^{*} = {(e_i,e_i)\mid i\in \llbracket 1,n\rrbracket}$ (réflexivité)

Pour $j = 1$ à $n$ faire

  Pour $i = 1$ à $n$ faire

    Pour $k = 1=$ à $=n=$ faire

      Si $(e_i, e_j)\in R^{*}$ et $(e_j, e_k)\in R^{*}$ et $(e_i,e_k) \not\in R^{*}$
      alors $R^{*} = R^{*}\cup \{(e_i,e_k)\}$


\paragraph{Clôtures d'un ensemble par des relations}

\begin{enumerate}
  \item Soient E un ensemble, $n \in \mathbb{N}^{*}$ et $R \subseteq E^{n+1}$
  une relation d'arité $n+1$ sur E.
  
  Un sous-ensemble $F \subseteq E$ est dit clos (fermé) relativement à $R$ si pour
  tout $(e_1, e_2, \dots, e_n) \in R$, $e_1 \in F$, $e_2 \in F$, ..., $e_n \in F$
  alors $e_{n+1} \in F$.

  \item Plus généralement, soit $E$ un ensemble et $R_1$, $\dots$, $R_k$ des relations
  sur E ($R_i \subseteq E^{n_i}$).

  Un sous-ensemble $F \subseteq E$ est dit clos (fermé) relativement aux relations 
  $R_1, \dots, R_k$ s'il est clos relativement à chacun des relations $R_i$ pour
  $1 \leq i \leq k$.
\end{enumerate}


\paragraph{Problème}
%%% EXO (?)

$E, R_1, \dots, R_k$ tels que $R_i \subseteq E^{n_i}$ pour $1 \leq i \leq k$

$E' \subseteq E$

Construire le plus petit ensemble $F$ contenant $E'$ tel que $F$ est clos
relativement à $R_1, \dots, R_k$.

$F$ est appelé la fermeture de $E'$ relativement à $R_1, R_2, \dots, R_k$

\paragraph{Propiété}
Soient $R_1,..., R_n$ des relations sur un ensemble $E$

% \subseteq ->
et soit $E' \subseteq E$. Alors il existe un unique sous-ensemble minimal, au 
sens de l'inclusion tel que $F'$ est inclus dans $F$ et $F$ est fermé (clôs) 
relativement aux relations $R_1, R_2,\dots, R_n$.

\paragraph{Preuve}

Comme $R_1, R_2, \dots, R_n$ sont des relations sur $E$, alors $E$ est clos relativement à
$R_1, R_2, \dots, R_n$.

$S$ est l'ensemble de tous les sous ensembles de $E$ qui contiennent $E'$ et qui sont
fermés relativement à $R_1, R_2, ..., R_n$
alors $ {A \subseteq E \mid E'\subseteq A}$ est fermé relativement à $R_1, R_2,\dots, R_n$

On a  $E \in S$ donc $S$ est non-vide.

Posons $B = \bigcup_{A\in S}$

On a donc : 
\begin{enumerate}
  \item $E \subseteq B$
  \item %CHECKME : (c'est bien un i?) 
  Si $(a1, a_i, a_{i+1})$ pour $1 \leq i \leq k$
  et si $a_1, ..., a_k \in B \text{ alors } a_{k+1} \in A$
\end{enumerate}

\paragraph{Algorithme : Clotûre}

Donnée : $E = {e_1, e_2, \dots, e_n}$
$R_1$ dans $E^{d_1+1}$, $R_2$ dans $E^{d_2+1}, \dots, E^{d_n + 1}$,

$E'$ dans $E$

Résultat : $F$ est la fermeture de $E'$ relativement aux relations $R_1, R_2, \dots, R_n$

$F=E'$

tant qu'il existe un $i$ de ${1, \dots, k}$ et un $(a_1, a_2, \dots , a_i)$

On a 
% FIXME

\paragraph{Exercice}

$R \subseteq E^2 -> R^{*}$

$R_1 = { (e,e) \mid (e,e) \in E^2}$
$R_2 = { ((e_1,e_2),(e_2,e_3),(e_1,e_3))  \in E^2 \times E^2 \times E^2 }$

%%%%
%%%% pas complet parce que c9 déconne :(
%%%% cette partie est sur le drive


\chapter{Alphabets, mots et langages}

\section{Définitions}

\subsection{Alphabet}

Un alphabet est un ensemble de $\Sigma$ de symboles (figures) non vide et fini.

Les élémentes de l'alphabet $\Sigma$ sont appelés des lettres de l'alphabet.

\paragraph{Exemples}

\begin{itemize}
\item $R = {A, B, C, ..., Z}$
\item $B = {0, 1}$
\end{itemize}

\subsection{Mot}
Un mot sur un alphabet sigma est une suite finie d'éléments de sigma.

$\mid m \mid$ est la longueur du mot $m$ sur l'alphabet $\Sigma$ c'est-à-dire le nombre d'éléments de
la suite qui compose le mot $m$.

\paragraph{Exemple}

$A, A, B, C, B, E$ est un mot de longueur $6$ sur l'alphabet $R$.

La suite vide (sans aucun élément) est appelée $e$.

Le mot vide est noté dans la suite $\epsilon$.

% TODO remplacer les |x| par le bon symbole
Si $m$ est un mot sur l'alphabet $\Sigma$ et si $1 < i \leq \mid m \mid$,
alors $m(i)$ est le $i$-eme élément de la suite $m$

Autrement dit $m = m(1), m(2), \dots, m(n)$.

\subsection{$\Sigma^{*}$}

Soit $\Sigma$ un alphabet, $\Sigma^{*}$ est l'ensemble de tous les mots de
l'alphabet $\Sigma$.

\paragraph{Exemple}

$B^{*} = { \epsilon, 0, 1, 00, 01, 10, 11, 000, \dots}$ est l'ensemble de tous
les mots de l'alphabet $B$. % cf. plus haut

\paragraph{Remarque}
$\Sigma^{*}$ est un ensemble infini dénombrable.

% bla bla pour n lettres on a sigma^n mots possibles

Un langage $L$ sur un alphabet $\Sigma$ est un sous ensemble de $\Sigma^{*}$.

% joli P, je me souviens plus comment on fait - ah ok
 $\mathcal{P} (\Sigma^{*}) $ est l'ensemble de tous les langages sur l'alphabet
 $\Sigma$.
 
 $\Sigma = \beta = {0,1}$
$L_1 = {}, L_2 = {\Sigma}$
$L_3 = {0,1,11}$
$L_4 = {0,00,01,000,001,010 ... }$

\subsection{}

Sur $\Sigma *$ on définit l'opération de concaténation de la façon suivante :

Soit $m' \in \Sigma^{*}$.

Si $m = a1, a2, ..., an$ et $m' = a'1, a'2, ..., a'n$,
alors $m'' = m . m' $ % .... ?

Autrement dit $ \mid m'' \mid  = \mid m \mid +  \mid m' \mid $
et m''(i) = { m(i) pour i <= i <|m|
m''(i - |m|) pour m+1 <= i <= |m| + |m'|
%% i have no idea what I'm writin' -lulz

\section{Propriétés}

Soit $\sigma$ un alphabet

1. Pour tout  m de $\Sigma^{*}$ on a :

$m . \epsilon = \sigma . m = m$
$\sigma$ est l'élément neutre de l'opération

2. pour tout m, m', m'' de $\sigma*{*}$on a : m . (m' . m'') = (m . m') . m''

. est opération associative

\section{Définitions}
% chaque 1. est un subsection

Soient m, m' de sigma *, a de sigma et i de |N

1. On dit que m' est un sous-mot de m s'il existe u, v de $\sigma^{*}$ tels que

\[
  m = u . m' . v
\]

2. On dit que m' est suffixe de m s'il existe u de $\sigma^{*}$ tel que
m = u . m' . % ?

3. On dit que m' est préfixe de m s'il existe v de sigma etoile tel que
m = m'.v

\section{Propriétés}

\subsection{~}

$m^0 = \epsilon$

$m^ = m^i . m$ pour i>=0

\subsection{Mot miroir}

% joli R (mais pas l'ensemble des réels)
%-JugglerMurdoc : je crois que Mathcab c'est pas dans l'ensemble des réels
%ce'est juste cursif
$\epsilon^{R} = \epsilon$

$(a .m)^{R} = m^{R} . a$

$m^R$ est appelé le 
% emphase
mot inverse
%
du mot m (ou /mot miroir/) 

$010101^R = 101010$

\section{Opérations sur les langages}

Soient $\sigma$ un alphabet et L1, L2 (L1, L2 inclus dans $\sigma^{*}$)
deux langages.

\subsection{Concaténation}

$L1 . L2 = {m1 . m2 | m1 de L1, m2 de L2}$

\paragraph{Exemple}

L1 = {01, 10}
L2 = {00, 10}
L1 . L2 = {0100, 0110, 1000, 1010}


\subsection{Opération de Kleene}

L2* = { m1 . m2 . mk | k de |N et m1, m2, ..., mk de L1}

\subsection{~}

$L_1^{+} = L_1^{*} \ {\epsilon}$


\section{Représentations finies des langages}

\subsection{Expression rationnelle}

Soit $\Sigma$ un alphabet.

% entourer les opérateurs pour mieux les distinguer
Posons $\Sigma_d = \Sigma \cup {\emptyset, *, (, ), \cup}$

\subsection{Définition}

Une expression rationnelle sur $\Sigma$ est un mot sur l'alphabet $\Sigma . d$
obtenu en respectant les règles suivantes.

\begin{enumerate}

  \item $\emptyset$ et $a \in \sigma$ sont des expressions régulières pour 
  tout $a \in \Sigma$
  
  \item Si $\alpha$ et $\beta$ sont des expressions régulières alors 
  $(\alpha \beta)$ est une expression régulière%--> Parenthèses entourées!!
  \item Si alpha et beta sont des expressions regulieres alors
  \[ 
    ( \alpha \cup \beta ) est une expression régulière
  \]
  \item Si $\alpha est une expression régulière alors \alpha * $ % * entourée
  est une expression régulière.
  \item Seuls les mots de $\Sigma^{*}_{d}$ qui sont construits en utilisant les
  règles $R_1, R_2, R_3 et R_4$ sont des expressions régulières.
\end{enumerate}

\section{Langage associé à une expression rationnelle}
Soit e une expression rationnelle sur un alphabet Sigma.

Le langage L(e) % joli L
décrit par l'expression rationnelle e est obtenu en interprétant les caractères
de $\Sigma_d$ de la façon suivante :
\begin{enumerate}
\item $L(\emptyset) = \emptyset = {}$
\item L(a)= { a} pour tout a $\in \Sigma$

% aucune idée des symboles utilisés ici autour de alpha et beta
%'Parenthèses avec barre à l'intérieur'
\item  L(alpha, beta) = L(alpha) . L(beta) pour alpha, beta expressions
rationelles

L(

\end{enumerate}
\paragraph{Exemple}
% bla bla bla bla bla ça m'énerve.
L(

\subsection
L $\subseteq \Sigma^{*}$ est dit langage régulier s'il eiste une expression 
régulière e tel que L = L(E)%Pas sûr...

\end{document}
