\documentclass[12pt, a4paper]{report}

\usepackage[utf8]{inputenc}
\usepackage[T1]{fontenc}
\usepackage[francais]{babel}
\usepackage{amsmath}
\usepackage{amssymb}
\usepackage{tikz}

\title{Théorie des langages}
\author{Licence d'informatique}
\date{Université de Strasbourg}

\begin{document}

\maketitle

\part{Langages}

\chapter{Introduction}

\section*{Quelques définitions}

\paragraph{Langage} On appelle \textit{langage} tout ensemble de mots. Il existe deux façons de définir un langage : par extension ou par intention Le mot vide est noté $\Sigma$

\paragraph{Alphabet} On appelle \textit{alphabet} tout ensemble fini (non vide) de symboles.


\paragraph{Grammaire} 

Pour les langages de programmation, un programme peut être vu comme un
mot faisant partie de ce langage. Le compilateur vérifie ensuite dire
que ce mot est conforme au langage, à ce qu’il attend. On met donc des
mécanismes d’analyse syntaxique, on parle de \textit{grammaire}.


\paragraph{Exemple} langage binaire

Alphabet : $\{0, 1\}$

Mots : 0, 1, 01, 00001.


\begin{center}
\begin{tabular}{|c|c|c|}
\hline
classes de langages & reconnaissance d'un mot & génération d'un mot \\ \hline \hline
langage régulier  & automates finis & grammaire régulière \\ \hline
langage algébrique & automates à pile & grammaire algébrique \\ \hline
langage récursif & machine de Turing & grammaires sans contrainte \\
\hline
\end{tabular}
\end{center}

Ces classes de langage sont présentées du plus restreint au plus large, ou plus formellement :
\[
\text{langages réguliers}
\;\subseteq\text{langages algébriques}
\;\;\subseteq\text{langages récursifs}
\;\;\;\subseteq\text{tous les langages}
\]

\section{Ensemble, relation, fonction et langage}


Un \textit{ensemble} est une collection non-ordonnée d'objets qui sont
appelés les \textit{éléments} de l'ensemble.

Les ensembles ont été formellement définis suite au \textit{paradoxe
  de Russel} .

\subsection{Exemple}

\[
I = {1, 2, 3, 5, 8, 10}
\]


\subsection{Opérations sur les ensembles}

Soient $A$ et $B$ deux ensembles.

\paragraph{Union}
\[ A \cup B = \{ e | e \in A \textrm{ ou } e \in B \} \]

\paragraph{Intersection}
\[ A \cap B = \{ e | e \in A \textrm{ et } e \in B \} \]

\paragraph{Différence}
\[ A \backslash B = \{ e | e \in A \textrm{ et } e \not\in B \} \]

\paragraph{Différence symétrique}
\[ A \Delta B = (A \backslash B)\cup (A \backslash B) \]

\paragraph{Ensemble des parties}
\[ P(A) = { C | C \subseteq A} \]

\paragraph{Produit cartésien}
\[ A \times B = {(a, b) | a \in A \text{ et } b \in B} \]

\subsection{Lois de Morgan}
Soient $A, B, C$ des ensembles. Alors
\[
A\backslash(B \cup C) = (A\backslash B=\cap(A\backslash C) \\
A\backslash(B \cap C) = (A\backslash B=\cup(A\backslash C)
\]

\paragraph{Remarques :} soient $A, B$ deux ensembles. Alors :
\[
A=B \Leftrightarrow A\subseteq B \text{ et } B\subseteq A
\]

Pour des $n$-uplets, l'égalité devient a :
\[
(e_1,e_2,\dots,e_n)=(f_1,f_2,\dots,f_n)\Leftrightarrow e_1=f_1,e_2=f_2,\dots,e_n=f_n
\]

\subsection{Définition}


Soit $A$ un ensemble et $\Pi \subseteq \mathcal{P}(A)$

$\Pi$ est dit une partition de $A$ si et seulement si :
\begin{itemize}
\item $A = \bigcup_{k\in K} \text{ ou }\Pi=\{B_k \mid k\in K\}$
\item $B_i\bigcap B_j = \emptyset \; \forall i, j\in K \mid i\neq j$
\item $B_i \neq \emptyset$
\end{itemize}

Exemple : $A=\{1,2,3,4,5\}, \Pi=\{\{1,3\},4,\{2,5\}\} = \{B_1,B_2,B_3\}$

\subsection{Relations}

Soient $E_1$, $E_2$, $\dots$, $E_n$, $n$ ensembles. Une relation $R$
$n$-aire (d'arité $n$) sur $E_1$, $E_2$, ..., $E_n$ est un
sous-ensemble de $E_1 \times E_2 \times \dots \times E_n$.


Autrement dit $R \subseteq E_1 \times E_2 \times \dots \times E_n$.


Pour $n = 2$, R est dite une relation binaire.

\paragraph{Notation} Si $E = E_1 = E_2 = \dots = E_n$, alors $E_1 \times \dots \times E_n$ est noté $E^{n}$.



\subsection{Relations fonctionelles}

Soit $R$ une relation $n$-aire sur $E_1\times E_2\times\dots
E_{n-1}\times E_n$ avec $n\geq 2$. Si, pour tout
$(e_1,e_2,\dots,e_{n-1})\in E_1\times E_2\times\dots E_{n-1}$, il
existe \textit{un et un seul} $e_n\in E_n$ tel que
$(e_1,e_2,\dots,e_{n-1},e_n) \in R$.

On dit alors que $R$ est une relation fonctionnelle relativement à
$E_1\times E_2\times\dots E_{n-1}$. On utilisera la notation : $e_n =
f_R(e_1,\dots,e_{n-1})$ qui est l'unique élément tel que
$(e_1,e_2,\dots,e_{n-1},e_n)\in R$.

De plus, $f_R$ est appelé la fonction associée à la relation
fonctionnelle $R$. $E_1\times E_2\times\dots E_{n-1}$ est appelé le
domaine (ou ensemble de départ) de $f_R$ et $E_n$ est appelé
l'ensemble d'arrivée de $f_R$.

\paragraph{Notation} soit $R$ une relation fonctionne sur $E_1\times\dots\times E_{n-1}\times E_n$. On note :
\[
f_R : E_1\times\dots\times E_{n-1} \longmapsto E_n
\]

\paragraph{Définitions} soient $A, B$ deux ensembles et $f_R : A\mapsto B$.

\begin{itemize}
\item $f_R$ est dite injective (une injection) si $\forall a, a' \in
  A$
\[a\neq a' \Rightarrow f_R(a) \neq f_R(a')\]

\item $f_R$ est dite surjective (une surjection) si $\forall b\in B$,
  il existe $a\in A$ tel que $b = f_R(a)$

\item $f_R$ est dite bijective (une bijection) si $f_R$ est à la fois
  injective est bijective.

\item Lorsque la fonction $f_R$ est bijective, alors $f_R^{-1}$ est la
  fonction associée à la relation $R^{-1} = \{(b,a) \mid (a,b) \in
  R\}$. En particulier, $R^{-1}$ est fonctionelle relativement à
  $B$. Autrement dit :
\[f_R^{-1} = f_{R^{-1}}\]
Dans ce cas, $f_{R^{-1}}$ est bijective.

\item De plus, $\forall (a, b) \in A\times B$ : 
\[b=f_R(a) \leftrightarrow a = f_R^{-1}(b)\]
Si $f_R$ est bijective, alors $\big((f_R)^{-1}\big)^{-1} = f_R$

\item Soient $R_1 \subseteq A\times B$ et $R_2 \subseteq B\times C$
  deux relations binaires.
\[R_1 \circ R_2 = \{(a,c) \in A \times C \mid \exists b\in B \text{ tel que }(a,b)\in R_1 \text{ et } (b,c)\in R_2\} \]
$R_1 \circ R_2$ est la composition de $R_1$ et de $R_2$.
\end{itemize}

\paragraph{Propriété :} soit $R$ une relation fonctionnelle sur $A\times B$ telle que $f_R$ est bijective. Alors :

\begin{itemize}
\item $R\circ R^{-1} = \{(a,a)\mid a\in A\}$
\item $R^{-1}\circ R = \{(b,b)\mid b\in B\}$
\end{itemize}

\subsection{Relations fonctionnelles particulières}
Soir $R$ une relation binaire sur $A^2 (R\subseteq A^2)$.

\begin{itemize}
\item $R$ est dite réflexive si $\forall a\in A, (a,a)\in R$

\item $R$ est dite symétrique si $\forall a, b\in A, (a,b)\in R
  \Rightarrow (b,a)\in R$

\item $R$ est dite transitive si $\forall a, b, c\in A, (a,b)\in R$ et
  $(b,c)\in R \Rightarrow (a,c)\in R$

\item $R$ est une relation d'équivalence si $R$ est à la fois
  réfléxive, symétrique et transitive.

\item $R$ est dite antisymétrique si $\forall a, b\in A, (a,b)\in R$
  et $a\neq b \Rightarrow (b,a)\not\in R$
\item $R$ est une relation d'ordre si $R$ est à la fois réflexive, antisymétrique et transitive.
\item $R$ est une relation d'ordre total si $R$ est une relation d'ordre et pour tout $a, b\in A$, si $(a,b) \not\in R$, alors $(b,a)\in R$
\end{itemize}

\subsection{Représentation de relations}

Soit $A = \{1,2,3\}$ un ensemble, et la relation $R = \{(1,2),(1,3),(2,3)\}$ une relation sur $A$. On peux représenter cette relation sous plusieurs formes :

\paragraph{Forme matricielle}
%\[
%M = \bordermatrix{~ & 1 & 2 & 3 \cr
%                  1 & 0 & 1 & 1 \cr
%                  2 & 0 & 0 & 1 \cr
%                  3 & 0 & 0 & 0 \cr}
%\]

\paragraph{Graphes}
% représentation en graphe
%R = (1,2),(1;3),(2,3)
%1 -> 2
%1 -> 3
%2 -> 3
%%% mux> ça compile pas, j'ai commenté
% \begin{tikzpicture}
% \tikzstyle{vertex}=[circle,fill=black!25,minimum size=20pt,inner sep=0pt]
% \tikzstyle{edge} = [draw,thick,->]
% \tikzstyle{weight} = [font=\small]
% \begin{figure}
% \begin{tikzpicture}[scale=1.8, auto,swap]
%    % draw the vertices
% \foreach \pos/\name in {{(0,2)/a}, {(3,2)/b}}
%     \node[vertex] (\name) at \pos {$\name$};
% % Connect vertices with edges and draw weights
% \foreach \source/ \dest /\weight in {a/b/1, b/a/2}
%     \path[edge] (\source) -- node[weight] {$\weight$} (\dest);
 
% \end{tikzpicture}
% \end{figure}


\paragraph{Notations}

Soit $R$ une relation d'équivalence $A \times A$ ($A^{2}$) et soit $a
\in A$.

\begin{itemize}

%%%% CHECKME vérifier les accolades
\item $[a]_{R} = { b \in A | (a,b) \in R} }$ est la classe de $A$
relative à la relation d'équivalence $R$

\item $A/R = {[a]_{R} | a \in A}$ est le quotient de $A$ par $R$
\end{itemize}

\paragraph{Propriétés}

\begin{itemize}

\item Soit $R$ une relation d'équivalence par $A \times A$. Alors
  $A/R$ est une partition de l'ensemble $A$.

\item Réciproquement si $\Pi$ est une partition de $A$ alors il existe
  une et une seule relation d'équivalence $R_{\Pi}$ tel que $\Pi =
  A/R_{\Pi}$.

\end{itemize}

\paragraph{Preuve :}
\begin{itemize}

\item $\forall a\in A, a\in [a]_R$ car $R$ est réflexive. Donc $A =
  \bigcup_{a\in A}[a]_R$

\item Soient $a, a\in A$. Montrons que si $[a]_R\cap [a']_R
  =\emptyset$, alors $[a]_R=[a']_R$
\end{itemize}

\paragraph{Relations binaires particulières}


Soit $b \in [a]_{R} \cap [a']_{R}$.


Alors $(a, b) \in R$ et $(a', b) \in R$


Comme $R$ est symétrique alors $(b, a') \in R$ et comme $R$ est
transitive alors $(a, a') \in R$.

Et donc $\forall c \in [a']_{R}$ on a $(a', c) \in R$ et donc par
transitivité $(a,c) \in R$ et donc $c \in [a]_{R}$.


\end{document}
